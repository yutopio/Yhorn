\documentclass[a4paper,12pt]{article}

\usepackage{amsmath}
\usepackage{algorithm}
\usepackage{algorithmic}
\usepackage{listings}
\usepackage{stmaryrd}
\usepackage{syntax}

\newcommand{\sembrack}[1]{[\![#1]\!}
    
\title{Algorithm explanation}
\author{Yuto Takei \\ The University of Tokyo}

\begin{document}
\maketitle

\section{Algorithm}

The Horn clause solving algorithm receives an input problem in Horn
graph. A Horn graph is a labelled directed graph $G=(V,E,\varphi)$
where V is a set of vertices and E is a set of edges. $\varphi: V
\rightarrow L$ is a labelling function for nodes, where L is a set of
elements of a predicate variable $P(x_1, ..., x_n)$ or a linear
inequality $a_1 x_1 + \cdots + a_n x_n \leq b$.

The meaning $\llbracket G \rrbracket $ of a Horn graph $G$ is a set of
Horn clauses $\mathcal{HC}_v$ for all vertices $v \in V$.

\begin{align*}
\mathcal{HC}_v &= \left( \bigwedge_{u \in pred(v)} \varphi(u) \right)
\Longrightarrow \varphi(v) \\ \llbracket G \rrbracket &= \bigwedge_{v
  \in V} \mathcal{HC}_v
\end{align*}

The solution of $G$ is a map $\rho$ from predicate variables to linear
predicates of the form $\lambda x_1, \cdots ,x_n. \psi $ (where $\psi$
is a formula constructed from
\begin{align*}
\psi ::= &a_1 x_1 + \cdots + a_n x_n \leq b \vert \\ &\psi \wedge \psi
\vert \psi \vee \psi \vert true \vert false
\end{align*}
), where $\rho[G]$ is a tautology. A solution is \textit{simple} if
$\rho[P]$ is of the form $a_1 x_1 + \cdots + a_n x_n \leq b$ for every
$P$ in $dom(\rho)$.

The problem of solving Horn clauses (represented by $G$) is to find a
solution of $G$ (and a simple solution if possible).

We consider below a Horn graph that is acyclic.

Given an acyclic Horn graph (or called a Horn DAG), we define the
relation $v <^+ v'$ by
\[ v <^+ v' \Longleftrightarrow (v,v') \in E \]
By the assumption that $G$ is acyclic, $<^+$ is a well-founded
relation.  We assume without loss of generality that $V$ contains the
least element $v_\bot$ with respect to $<^+$; otherwise we can always
ignore $v_\bot$ with $\varphi(v_\bot) = 0 \leq 0$.

For each $v \in V$, we define $\Delta(v) = (a_v x \leq b_v, C_v)$,
where
\begin{itemize}
\item $a_v x \leq b_v$ is a linear inequality that may contain
  coefficient variables
\item $C_v$ is a set of linear expressions on coefficient variables.
\end{itemize}
Intuitively, the pair $(\mathbf{a} \mathbf{x} \leq b, C)$ denotes the
set of inequalities:

\begin{align*}
\left\lbrace
 \mathbf{a} \mathbf{x} \leq b |
 \exists \left( FV(C)
  \setminus (\mathbf{a} \cup \left\lbrace b \right\rbrace
 \right). C
\right\rbrace
\end{align*}

$\Delta(v)$ is defined by well-founded induction on $v$ with respect
to the relation $<^+$. Then,
$\Delta(v) = \left( \mathbf{a}_v \mathbf{x} \leq b_v, C_v \right)$,
where

\begin{align*}
C_v =
\begin{cases}
\left\lbrace \mathbf{a}_v = \sum_{(u,v) \in E} \mathbf{a}_u \right\rbrace \cup
\left\lbrace b_v >= \sum_{(u,v) \in E} b_u \right\rbrace \cup
\bigcup_{(u,v) \in E} C_u
& \mbox{if } \varphi(v) = P(\mathbf{x}) \vee v = v_\bot \\
\left\lbrace \mathbf{a}_v = \lambda_v \mathbf{a}_0 \right\rbrace \cup
\left\lbrace b_v = \lambda_v b_0 \right\rbrace
& \mbox{if } \varphi(v) = \mathbf{a}_0 \mathbf{x} \leq b_0
\end{cases}
\end{align*}

Given $\varphi(v_\bot) = \mathbf{a}_0 \mathbf{x} \leq b_0$, any
solution $\sigma$ to the constraint $C_\star = C_{v_\bot} \cup {
  \mathbf{a}_0 = \mathbf{a}_{v_\bot} } \cup { b_0 = b_{v_\bot} }$
gives a simple solution for $G$ if $C_\star$ is satisfiable, i.e., we
should have:

\quote
If $\models \sigma(C_\star)$, then $\rho$ defined by
\begin{align*}
 \rho = \left\lbrace
  \left( P, \sigma(\mathbf{a}_v \mathbf{x} \leq b_v) \right) |
  \forall v \in V. \varphi(v) = P(\mathbf{x})
 \right\rbrace
\end{align*}
is a solution of $G$.
\endquote

\end{document}
