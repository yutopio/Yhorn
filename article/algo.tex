\documentclass[a4paper,12pt]{article}

\usepackage{amsmath}
\usepackage{algorithm}
\usepackage{algorithmic}
\usepackage{listings}
\usepackage{stmaryrd}
\usepackage{syntax}

\newcommand{\sembrack}[1]{[\![#1]\!}
    
\title{Algorithm explanation}
\author{Yuto Takei \\ The University of Tokyo}

\begin{document}
\maketitle

\section{Algorithm}

The Horn clause solving algorithm receives an input problem in Horn
graph. A Horn graph is a labelled directed graph $G=(V,E,\varphi)$
where V is a set of vertices and E is a set of edges. $\varphi: V
\rightarrow L$ is a labelling function for nodes, where L is a set of
elements of a predicate variable $P(x_1, ..., x_n)$ or a linear
inequality $a_1 x_1 + \cdots + a_n x_n \leq b$.

The meaning $\llbracket G \rrbracket $ of a Horn graph $G$ is a set of
Horn clauses $\mathcal{HC}_v$ for all vertices $v \in V$.

\begin{align*}
\mathcal{HC}_v &= \left( \bigwedge_{u \in pred(v)} \varphi(u) \right)
\Longrightarrow \varphi(v) \\ \llbracket G \rrbracket &= \bigwedge_{v
  \in V} \mathcal{HC}_v
\end{align*}

The solution of $G$ is a map $\rho$ from predicate variables to linear
predicates of the form $\lambda x_1, \cdots ,x_n. \psi $ (where $\psi$
is a formula constructed from
\begin{align*}
\psi ::= &a_1 x_1 + \cdots + a_n x_n \leq b \vert \\ &\psi \wedge \psi
\vert \psi \vee \psi \vert true \vert false
\end{align*}
), where $\rho[G]$ is a tautology. A solution is \textit{simple} if
$\rho[P]$ is of the form $a_1 x_1 + \cdots + a_n x_n \leq b$ for every
$P$ in $dom(\rho)$.

The problem of solving Horn clauses (represented by $G$) is to find a
solution of $G$ (and a simple solution if possible).

We consider below a Horn graph that is acyclic.

Given an acyclic Horn graph (or called a Horn DAG), we define the
relation $v <^+ v'$ by
\[ v <^+ v' \Longleftrightarrow (v,v') \in E \]
By the assumption that $G$ is acyclic, $<^+$ is a well-founded
relation.  We assume without loss of generality that $V$ contains the
least element $v_\bot$ with respect to $<^+$; otherwise we can always
ignore $v_\bot$ with $\varphi(v_\bot) = true$.

For each $v \in V$, we define $\Delta(v) = (a_v x \leq b_v, C_v)$,
where
\begin{itemize}
\item $a_v x \leq b_v$ is a linear inequality that may contain
  coefficient variables
\item $C_v$ is a set of linear inequalities on coefficient variables.
\end{itemize}
Intuitively, the pair $(\mathbf{a} \mathbf{x} \leq b, C)$ denotes the
set of inequalities:

\[ \left\lbrace \mathbf{a} \mathbf{x} \leq b | \exists \neg a,\neg
b. C \right\rbrace \]

(where $\exists \neg a, \neg b.$ is the existential quantification
over coefficient variables other than $a$ and $b$.

$\Delta(v)$ is defined by well-founded induction on $v$ with respect
to the relation $<^+$.

Then,

\begin{align*}
\Delta(v) =
\begin{cases}
\begin{split}
\Bigg(
\sum_i { a_i a_{0,i} x_i } \leq b b_0,
\bigwedge_i \left( - a_i a_{0,i} + \sum_{\substack{(u,v) \in E \\ \Delta(u)=\mathbf{a}_u \mathbf{x} \leq b_u}} a_{u,i} = 0 \right) \wedge \\
\left( \sum_{\substack{(u,v) \in E \\ \Delta(u)=\mathbf{a}_u \mathbf{x} \leq b_u}} b_u - b (b_0+1) < 0 \right) \wedge \\
\bigwedge_{(u,v) \in E} C_u
\Bigg)
& \mbox{if } \varphi(v) = \mathbf{a}_v \mathbf{x} \leq b_v
\end{split}
 \\
\begin{split}
\Bigg(
\mathbf{a} \mathbf{x} \leq b,
\bigwedge_i \left( - a_i + \sum_{\substack{(u,v) \in E \\ \Delta(u)=\mathbf{a}_u \mathbf{x} \leq b_u}} a_{u,i} = 0 \right) \wedge \\
\left( \sum_{\substack{(u,v) \in E \\ \Delta(u)=\mathbf{a}_u \mathbf{x} \leq b_u}} b_u - (b+1) < 0 \right) \wedge \\
\bigwedge_{(u,v) \in E} C_u
\Bigg)
& \mbox{if } \varphi(v) = P(\mathbf{x}).
\end{split}
\end{cases}
\end{align*}

where $\mathbf{a}$ and $b$ are fresh coefficient variables.

If $C_{v_\bot}$ is satisfiable, then any solution $\sigma$ for it
gives a simple solution for $G$, i.e., we should have:

If $\models \sigma(C_{v_\bot})$, then $\rho$ defined by
\[ \rho(v)= \sigma(a_v x <= b_v) \]
is a solution of $G$.

\end{document}
