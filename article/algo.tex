\documentclass[a4paper,12pt]{article}

\usepackage{amsmath}
\usepackage{amssymb}
\usepackage{algorithm}
\usepackage{algorithmic}
\usepackage{listings}
\usepackage{stmaryrd}
\usepackage{syntax}

\newcommand{\sembrack}[1]{[\![#1]\!}
\newcommand{\edge}[2]{#1\rightarrow#2}
\newcommand{\path}[2]{#1\rightarrow^\star#2}

\title{Algorithm explanation}
\author{Yuto Takei \\ The University of Tokyo}

\begin{document}
\maketitle

\section{Algorithm}

The Horn clause solving algorithm receives an input problem in Horn
graph. A \textbf{Horn graph} is a labelled directed graph $G=(V,E,\varphi)$.
\begin{itemize}
\item $V$ is a union of two sets of vertices; Horn term vertices $V_T$
  and arrow vertices $V_\Rightarrow$. Each vertex $u \in
  V_\Rightarrow$ has eactly one outgoing edge to some $v \in V_T$,
  denoted as $succ(u)$.
\item $E$ is a set of labelled edges, which are expressed in the form
  $(u,v,\theta)$ where $\theta$ is a finite map from variables to
  variables.
\item $\varphi: V_T \rightarrow L$ is a labelling function for Horn
  term vertices, where L is a set of elements of a predicate variable
  in the form $P(x_1, ..., x_{arity(P)})$ or a linear inequality $a_1
  x_1 + \cdots + a_n x_n \leq b$.
\end{itemize}

There are two kinds of edges in a Horn graph;
\begin{itemize}
\item for $u \in V_T, v \in V_\Rightarrow$, an edge takes a form
  $(u,v,\theta)$, or,
\item for $u \in V_\Rightarrow, v \in V_T$, an edge takes a form
  $(u,v,\emptyset)$, where the mapping is empty.
\end{itemize}

The meaning $\llbracket G \rrbracket $ of a Horn graph $G$ is a set of
Horn clauses $\mathcal{HC}_v$ for all vertices $v \in V_\Rightarrow$.

\begin{align*}
\mathcal{HC}_v & = \left( \bigwedge_{(u,v,\theta) \in E} \theta \varphi(u) \right) \Longrightarrow \varphi(succ(v)) \\
\llbracket G \rrbracket & = \bigwedge_{v \in V_\Rightarrow} \mathcal{HC}_v
\end{align*}

We implicitly assume each $\mathcal{HC}_v$ is universally quantified
by its all free variables $\textsc{FV}(\mathcal{HC}_v)$. Without loss
of generality, we limit occurrence of linear expression on right-hand
side of Horn clauses to either $0 \leq -1$, meaning $\bot$, or $0 \leq
0$, meaning $\top$.

Note that we allow multiple appearance of Horn clauses with the same
predicate variable on their right-hand side.

The solution of $G$ is a map $\rho$ from predicate variables to linear
predicates of the form $\lambda x_1, \cdots ,x_n. \psi $ (where $\psi$
is a formula constructed from
\begin{align*}
\psi ::= & a_1 x_1 + \cdots + a_n x_n \leq b \mid \\
& \psi \wedge \psi \mid \psi \vee \psi \mid \top \mid \bot
\end{align*}
), where $\rho[G]$ is a tautology. A solution is \textit{simple} if
$\rho[P]$ is of the form $a_1 x_1 + \cdots + a_n x_n \leq b$ for every
$P$ in $dom(\rho)$.

The problem of solving Horn clauses (represented by $G$) is to find a
solution of $G$ (and a simple solution if possible).

We consider below a Horn graph that is acyclic.

For simplicity of discussion, we define $k_v$ the arity of the
right-hand side of $\mathcal{HC}_v$. Assuming that variables in every
Horn clauses are appropriately renamed, the right-hand side of
$\mathcal{HC}_v$ takes the form $P(x_1,...,x_{k_v})$ where $k_v =
arity(P)$, or either $\top$ or $\bot$ with $k_v = 0$. Also we let $K$
denote the maximum of $k_v$ over all $v$.

Given an acyclic Horn graph (or called a Horn DAG), we define the
relation $u \xrightarrow{\theta} v$ over $V_T$ as follows.
\[ u \xrightarrow{\theta} v \Longleftrightarrow
\exists a \in V_\Rightarrow; \left\lbrace (u,a,\theta), (a,v,\emptyset) \right\rbrace \subseteq E \]
We may omit $\theta$ and write $u \rightarrow v$ for convenience for
later discussion. By the assumption that $G$ is acyclic,
$\rightarrow^+$ is a well-founded relation. We assume without loss of
generality that $V_T$ contains the greatest element $v_\bot$ with
respect to $\rightarrow^+$; otherwise we can always implicitly assume
$\varphi(v_\bot) = \top$ such that $\forall v \in V_T; v
\xrightarrow{\emptyset} v_\bot$.

For each $v \in V_T$, we define
$\Delta(v) = (\mathbf{a}_v \mathbf{x}_v \leq b_v, C_v)$
, where
\begin{itemize}
\item $\mathbf{a}_v \mathbf{x}_v \leq b_v$ is a linear inequality that
  contains coefficient variables $\mathbf{a}_v$, and
\item $C_v$ is a set of linear expressions on coefficient variables.
\end{itemize}
Intuitively, the pair $(\mathbf{a} \mathbf{x} \leq b, C)$ denotes the
set of inequalities:

\begin{align*}
\left\lbrace
 \mathbf{a} \mathbf{x} \leq b \middle|
 \exists \left( \textsc{FV}(C)
  \setminus (\mathbf{a} \cup \left\lbrace b \right\rbrace
 \right); C
\right\rbrace
\end{align*}

$\Delta(v)$ is defined by well-founded induction on $v$ with respect
to the relation $\rightarrow^+$. Then, $C_v$ in
$\Delta(v) = \left( \mathbf{a}_v \mathbf{x}_v \leq b_v, C_v \right)$
can be shown as follows.

\begin{align*}
C_v = & \bigcup_{u \rightarrow v} C_u
\\
& \cup \begin{cases}
\emptyset
& \mbox{if } \nexists u; u \rightarrow v \\
\left\lbrace
 \mathbf{a}_{v,i} = \sum_{u \xrightarrow{\theta} v} \mathbf{a}_{u, \theta^{-1} (i)},
 b_v \geq \sum_{u \xrightarrow{\theta} v} b_u
\right\rbrace \cup \\
\hspace{2cm} \bigcup_{k_v < i \leq K} \left\lbrace \mathbf{a}_{v,i} = 0 \right\rbrace
& \mbox{otherwise}
\end{cases}
\\
& \cup \begin{cases}
\emptyset
& \mbox{if } \varphi(v) = P(\mathbf{x}_v) \\
\left\lbrace
 \lambda_v \geq 0, \mathbf{a}_v = \lambda_v \mathbf{a}_0,
 b_v = \lambda_v b_0
\right\rbrace
& \mbox{if } \varphi(v) = \mathbf{a}_0 \mathbf{x}_v \leq b_0
\end{cases}
\end{align*}

For simplicity, we assume variables $\mathbf{a}_u$, $b_u$ are obtained
from $\Delta(u)$.

Any model $\sigma$ to the constraint $C_{v_\bot}$ gives a simple
solution for $G$ if $C_{v_\bot}$ is satisfiable, i.e., we should have:

\begin{quote}
If $\models \sigma(C_{v_\bot})$, then $\rho$ defined by
\begin{align*}
 \rho = \left\lbrace
  \left( P, \sigma(\mathbf{a}_v \mathbf{x}_v \leq b_v) \right) \middle|
  \forall v \in V_T; \varphi(v) = P(\mathbf{x}_v)
 \right\rbrace
\end{align*}
is a solution of $G$.
\end{quote}

If $C_{v_\bot}$ is unsatisfiable, no simple solution exists for
$G$. We now try to give a solution by relaxing the constraint
$C_{v_\bot}$.

Let $\hat V_T$ a set of Horn term vertices $V_T$ with multiple
successors.
\[ \hat V_T = \left\lbrace v \in V_T \middle|
\exists u, u', \theta, \theta';
\left\lbrace (v,u,\theta), (v,u',\theta') \right\rbrace \subseteq E \wedge
(u \ne u' \vee \theta \ne \theta') \right\rbrace \]
We define \textbf{Copy tree}, which is a tree $T=(V_C,E_C,\xi)$
rooted at $v_{v_\bot}$.
\begin{itemize}
\item $V_C$ is a set of nodes.
\item $E_C$ is a set of labelled edges, which are expressed in the
  form $(u,v,e)$, where $e \in E \cup {\bot}$.
\item $\xi: V_C \rightarrow \hat V_T \cup \left\lbrace v_\bot
  \right\rbrace$ is a labelling function for nodes. The root node
  $v_{v_\bot}$ has a label $v_\bot$.
\end{itemize}

We provide $\textsc{Copy}_v$ procedure to rename all free variables
for avoiding name collision.
\[ \textsc{Copy}_v (x_V) = x_{V \cup \left\lbrace v \right\rbrace} \]
All variables are then distinguished by the label of copy origin
nodes. For consistency, the variables without the label should be
interpreted as that it has $\emptyset$ for origin nodes.

We give a constraint function over a copy tree
$\hat \Delta(v,e)$ for $v \in V_C, e \in E$ as
\[ \hat \Delta (v,e) = \textsc{Copy}_v (\Delta_{V'} (\xi (v)))
\text{ such that } V' = \left\lbrace v' \middle| (v, v',e) \in E_S \right\rbrace \]
extending $\Delta$ function to
$\Delta_V (v) = (\mathbf{a}_{v,\emptyset} \mathbf{x}_v \leq b_{v,\emptyset}, C_v)$
as
\begin{align*}
C_v = & \bigcup_\circ C_u \cup \bigcup_\bullet C_u \cup
\\
& \cup \begin{cases}
\emptyset
& \mbox{if } \nexists u; u \rightarrow v \\
\left\lbrace
 \mathbf{a}_{v,\emptyset,i} =
  \sum_\circ \mathbf{a}_{u,U,\theta^{-1}(i)} +
  \sum_\bullet \mathbf{a}_{u,U, \theta^{-1}(i)}
\right\rbrace \cup \\
\hspace{2cm} \left\lbrace
 b_{v,\emptyset} \geq
  \sum_\circ b_{u,U} +
  \sum_\bullet b_{u,U}
\right\rbrace \cup \\
\hspace{2cm} \bigcup_{k_v < i \leq K} \left\lbrace \mathbf{a}_{v,\emptyset,i} = 0 \right\rbrace
& \mbox{otherwise}
\end{cases}
\\
& \cup \begin{cases}
\emptyset
& \mbox{if } \varphi(v) = P(\mathbf{x}_v) \\
\left\lbrace
 \lambda_v \geq 0, \mathbf{a}_{v,\emptyset} = \lambda_v \mathbf{a}_0,
 b_{v,\emptyset} = \lambda_v b_0
\right\rbrace
& \mbox{if } \varphi(v) = \mathbf{a}_0 \mathbf{x}_v \leq b_0
\end{cases}
\end{align*}
under
\[
\circ = \substack{
  u \rightarrow v \\
  \nexists u' \in V; \xi (u') = u \\
  \Delta_V (u) = (\mathbf{a}_{u,U} \mathbf{x}_u \leq b_{u,U}, C_u)}
\qquad \bullet = \substack{
  u \rightarrow v \\
  \exists u' \in V; \xi (u') = u \\
  \hat \Delta (u', u \rightarrow v) = (\mathbf{a}_{u,U} \mathbf{x}_u \leq b_{u,U}, C_u)} \]

This intuitively means that constraints are duplicated according to
the shape of copy tree $T$.
Given the constraint for the root node in the copy tree as
\[ \hat \Delta (v_{v_\bot}) = (\mathbf{a}_0 \mathbf{x}_{v_\bot} \leq b_0, C_{v_\bot}) \]
a model to $C_{v_\bot}$ gives a solution $\rho$ to $G$, if one exists.

\begin{align*}
 \rho = \left\lbrace
  \left( P, \bigwedge_S \sigma(\mathbf{a}_{v,S} \mathbf{x}_v \leq b_{v,S}) \right) \middle|
  \forall v \in V_T; \varphi(v) = P(\mathbf{x}_v)
 \right\rbrace
\end{align*}

If $C_{v_\bot}$ stays unsatisfiable, we let $T$ grow based on the
unsatisfiable core.

\end{document}
