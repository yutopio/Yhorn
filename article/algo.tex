\documentclass[a4paper,12pt]{article}

\usepackage{amsmath}
\usepackage{amssymb}
\usepackage{algorithm}
\usepackage{algorithmic}
\usepackage{listings}
\usepackage{stmaryrd}
\usepackage{syntax}

\newcommand{\sembrack}[1]{[\![#1]\!}
\newcommand{\edge}[2]{#1\rightarrow#2}
\newcommand{\edgel}[3]{#1\xrightarrow{#2}#3}

\title{Algorithm explanation}
\author{Yuto Takei \\ The University of Tokyo}

\begin{document}
\maketitle

\section{Algorithm}

\subsection{Preliminaries}

The Horn clause solving algorithm receives an input problem in Horn
graph. A \textbf{Horn graph} is a labelled directed graph $G=(V,E,\varphi)$.
\begin{itemize}
\item $V$ is a union of two disjoint sets of vertices; Horn term
  vertices $V_T$ and arrow vertices $V_\Rightarrow$. Each vertex $u
  \in V_\Rightarrow$ has eactly one outgoing edge to some $v \in V_T$,
  denoted as $succ(u)$.
\item $E$ is a set of labelled edges, which are expressed in the form
  $(u,v,\theta)$ where $\theta$ is a finite map from variables to
  variables.
\item $\varphi: V_T \rightarrow L$ is a labelling function for Horn
  term vertices, where L is a set of elements of a predicate variable
  in the form $P(x_1, \ldots, x_{\mathrm{arity}(P)})$ or a linear
  inequality $a_1 x_1 + \cdots + a_n x_n \leq b$.
\end{itemize}
There are two kinds of edges in a Horn graph;
\begin{itemize}
\item for $u \in V_T, v \in V_\Rightarrow$, an edge takes a form
  $(u,v,\theta)$, written $\edgel{u}{\theta}{v}$, or,
\item for $u \in V_\Rightarrow, v \in V_T$, an edge takes a form
  $(u,v,\emptyset)$, written $\edge{u}{v}$, where the mapping is
  empty.
\end{itemize}

The meaning $\llbracket G \rrbracket $ of a Horn graph $G$ is a set of
Horn clauses $\mathcal{HC}_v$ for all vertices $v \in V_\Rightarrow$.

\begin{align*}
\mathcal{HC}_v & = \left( \bigwedge_{(u,v,\theta) \in E} \theta \varphi(u) \right) \Longrightarrow \varphi(succ(v)) \\
\llbracket G \rrbracket & = \bigwedge_{v \in V_\Rightarrow} \mathcal{HC}_v
\end{align*}

We implicitly assume each $\mathcal{HC}_v$ is universally quantified
by its all free variables $\textsc{FV}(\mathcal{HC}_v)$.  Without loss
of generality, we limit linear expression labels on $V_T$ vertices
with incoming edges to either $0 \leq -1$, meaning $\bot$, or $0 \leq
0$, meaning $\top$.

The solution of $G$ is a map $\rho$ from predicate variables to linear
predicates of the form $\lambda x_1, \cdots ,x_n. \psi $ (where $\psi$
is a formula constructed from
\begin{align*}
\psi ::= & a_1 x_1 + \cdots + a_n x_n \leq b \mid \\
& \psi \wedge \psi \mid \psi \vee \psi \mid \top \mid \bot
\end{align*}
), where $\rho[G]$ is a tautology. A solution is \textit{simple} if
$\rho[P]$ is of the form $a_1 x_1 + \cdots + a_n x_n \leq b$ for every
$P$ in $\mathrm{dom}(\rho)$.

The problem of solving Horn clauses (represented by $G$) is to find a
solution of $G$ (and a simple solution if possible).

\subsection{Simple solution}

We consider below a Horn graph that is acyclic.

For simplicity of discussion, we define \textbf{term arity} $k_v$ for
every Horn term vertex $v \in V_T$ with incoming edges.
\begin{align*}
k_v =
\begin{cases}
\mathrm{arity}(P) & \mbox{if } \varphi(v) = P(x_1,...,x_{\mathrm{arity}(P)}) \\
0 & \mbox{if } \varphi(v) = \top \mbox { or } \varphi(v) = \bot
\end{cases}
\end{align*}
We let $K$ denote the maximum of $k_v$ over all $v \in V_T$.

We define the relation $\leadsto$ over $V_T$ as follows.
\[ u \mathop{\leadsto}^\theta v \Longleftrightarrow
\exists a \in V_\Rightarrow; \edgel{v}{\theta}{a} \wedge \edge{a}{v} \]
We may omit $\theta$ and write $u \leadsto v$ for convenience in later
discussions.

By the assumption that $G$ is acyclic, $\leadsto^+$ is a well-founded
relation. We assume without loss of generality that $V_T$ contains the
greatest element $v_\bot$ with respect to $\leadsto^+$; otherwise we
can always implicitly assume $\varphi(v_\bot) = \top$ such that
$\forall v \in V_T \setminus \{v_\bot\}; v \mathop{\leadsto}^\emptyset v_\bot$.

For each $v \in V_T$, we define
$\Delta(v) = (\mathbf{a}_v \mathbf{x}_v \leq b_v, C_v)$
, where
\begin{itemize}
\item $\mathbf{a}_v \mathbf{x}_v \leq b_v$ is a linear inequality that
  contains coefficient variables $\mathbf{a}_v$, and
\item $C_v$ is a set of linear expressions on coefficient variables.
\end{itemize}
Intuitively, the pair $(\mathbf{a} \mathbf{x} \leq b, C)$ denotes the
set of inequalities:
\begin{align*}
\left\lbrace
 \mathbf{a} \mathbf{x} \leq b \middle|
 \exists \left( \textsc{FV}(C)
  \setminus (\mathbf{a} \cup \left\lbrace b \right\rbrace
 \right); C
\right\rbrace
\end{align*}

$\Delta(v)$ is defined by well-founded induction on $v$ with respect
to the relation $\leadsto^+$. Then, $C_v$ in
$\Delta(v) = \left( \mathbf{a}_v \mathbf{x}_v \leq b_v, C_v \right)$
can be shown as follows.

\begin{align*}
\hat C_{a,v} = &
 \bigcup_{\edgel{u}{\theta}{a}} C_u \cup
 \bigcup_{0 < i \leq K}
 \left\lbrace
  \mathbf{a}_{v,i} = \sum_{\edgel{u}{\theta}{a}} \mathbf{a}_{u, \theta^{-1} (i)}
 \right\rbrace \\
 & \hspace{2cm} \cup
 \left\lbrace
  b_v \geq \sum_{\edgel{u}{\theta}{a}} b_u
 \right\rbrace \cup
 \bigcup_{k_v < i \leq K}
 \left\lbrace \mathbf{a}_{v,i} = 0 \right\rbrace
\\
C_v = & \bigcup_{\edge{a}{v}} \hat C_{a,v} \cup
\begin{cases}
\emptyset
& \mbox{if } \varphi(v) = P(\mathbf{x}_v) \\
\left\lbrace
 \lambda_v \geq 0, \mathbf{a}_v = \lambda_v \mathbf{a}_0,
 b_v = \lambda_v b_0
\right\rbrace
& \mbox{if } \varphi(v) = \mathbf{a}_0 \mathbf{x}_v \leq b_0
\end{cases}
\end{align*}

For simplicity, we assume variables $\mathbf{a}_u$, $b_u$ are obtained
from $\Delta(u)$.

Any model $\sigma$ to the constraint $C_{v_\bot}$ gives a simple
solution for $G$ if $C_{v_\bot}$ is satisfiable, i.e., we should have:

\begin{quote}
If $\models \sigma(C_{v_\bot})$, then $\rho$ defined by
\begin{align*}
 \rho = \left\lbrace
  \left( P, \sigma(\mathbf{a}_v \mathbf{x}_v \leq b_v) \right) \middle|
  \forall v \in V_T; \varphi(v) = P(\mathbf{x}_v)
 \right\rbrace
\end{align*}
is a solution of $G$.
\end{quote}

\subsection{Disjunction-free problems}

If $C_{v_\bot}$ is unsatisfiable, no simple solution exists for
$G$. We now try to give a solution by relaxing the constraint
$C_{v_\bot}$.

We consider below a Horn graph whose Horn term vertices has at
most one successor.

\subsubsection{Quantifier elimination}

A set of eliminating quantifiers are obtained as $quant(n)$.

\begin{align*}
in(n) & = \bigcup_{p\in pred(n)} out(p) \\
out(n) & = gen(n) \cup in(n) \\
dup(n) & = \left\lbrace t \middle| p,q \in pred(n); p \ne q \wedge
 t \in out(p) \wedge t \in out(q) \right\rbrace \\
quant(n) & = gen(n) \cup dup(n) \cup
 (out(n) \cap \bigcup_{s \in succ(n)} quant(s))
\end{align*}

Let $\hat V_T$ a set of Horn term vertices $V_T$ with multiple
successors.
\[ \hat V_T = \left\lbrace v \in V_T \middle|
\exists u, u', \theta, \theta';
\edgel{v}{\theta}{u} \wedge \edgel{v}{\theta'}{u'} \wedge
(u \ne u' \vee \theta \ne \theta') \right\rbrace \]
We define \textbf{Copy tree}, which is a tree $T=(V_C,E_C,\xi)$
rooted at $v_{v_\bot}$.
\begin{itemize}
\item $V_C$ is a set of nodes.
\item $E_C$ is a set of labelled edges, which are expressed in the
  form $(u,v,e)$, where $e \in E \cup {\bot}$.
\item $\xi: V_C \rightarrow \hat V_T \cup \left\lbrace v_\bot
  \right\rbrace$ is a labelling function for nodes. The root node
  $v_{v_\bot} \in V_C$ has a label $v_\bot$.
\end{itemize}

We provide $\textsc{Copy}_v$ procedure to rename all free variables
for avoiding name collision.
\[ \textsc{Copy}_v (x_V) = x_{V \cup \left\lbrace v \right\rbrace} \]
All variables with the same name are then distinguished by the label
of copy origin nodes $V$. For consistency, the variables without the
label should be interpreted as that it has $\emptyset$ for origin
nodes.

We give a constraint function over a copy tree
$\hat \Delta(v,e)$ for $v \in V_C, e \in E$ as
\[ \hat \Delta (v,e) = \textsc{Copy}_v (\Delta_{V'} (\xi (v)))
\text{ such that } V' = \left\lbrace v' \middle| (v, v',e) \in E_C \right\rbrace \]
extending $\Delta$ function to
$\Delta_{V'} (v) = (\mathbf{a}_{v,\emptyset} \mathbf{x}_v \leq b_{v,\emptyset}, C_v)$
as

\begin{align*}
\hat C_a = &
 \bigcup_{\edgel{u}{\theta}{a}} C_u \cup
 \bigcup_{0 < i \leq K}
 \left\lbrace
  \mathbf{a}_{v,i} = \sum_{\edgel{u}{\theta}{a}} \mathbf{a}_{u, \theta^{-1} (i)}
 \right\rbrace \\
 & \hspace{2cm} \cup
 \left\lbrace
  b_v \geq \sum_{\edgel{u}{\theta}{a}} b_u
 \right\rbrace \cup
 \bigcup_{k_v < i \leq K}
 \left\lbrace \mathbf{a}_{v,i} = 0 \right\rbrace
\\
C_v = & \bigcup_{\edge{a}{v}} \hat C_a \cup
\begin{cases}
\emptyset
& \mbox{if } \varphi(v) = P(\mathbf{x}_v) \\
\left\lbrace
 \lambda_v \geq 0, \mathbf{a}_{v,\emptyset} = \lambda_v \mathbf{a}_0,
 b_{v,\emptyset} = \lambda_v b_0
\right\rbrace
& \mbox{if } \varphi(v) = \mathbf{a}_0 \mathbf{x}_v \leq b_0
\end{cases}
\end{align*}

---

\begin{align*}
C_v = & \bigcup_{(u,U,\theta,C_u) \in S_v^-} C_u \cup
\bigcup_{(u,U,\theta,C_u) \in S_v^+} C_u
\\
& \cup
\left\lbrace
 \mathbf{a}_{v,\emptyset,i} =
  \sum_{(u,U,\theta,C_u) \in S_v^-} \mathbf{a}_{u,U,\theta^{-1}(i)} +
  \sum_{(u,U,\theta,C_u) \in S_v^+} \mathbf{a}_{u,U, \theta^{-1}(i)}
\right\rbrace \\
& \cup \left\lbrace
 b_{v,\emptyset} \geq
  \sum_{(u,U,\theta,C_u) \in S_v^-} b_{u,U} +
  \sum_{(u,U,\theta,C_u) \in S_v^+} b_{u,U}
\right\rbrace \\
& \cup \bigcup_{k_v < i \leq K} \left\lbrace \mathbf{a}_{v,\emptyset,i} = 0 \right\rbrace
\end{align*}
such that
\begin{align*}
& S_v^- = \left\lbrace (u, U, \theta, C_u) \middle|
  u \mathop{\leadsto}^\theta v \wedge
  \left( \nexists u' \in V'; \xi (u') = u \right) \wedge
  \Delta_{V'} (u) = (\mathbf{a}_{u,U} \mathbf{x}_u \leq b_{u,U}, C_u) \right\rbrace
\\
& S_v^+ = \left\lbrace (u, U, \theta, C_u) \middle|
  u \mathop{\leadsto}^\theta v \wedge
  \left( \exists u' \in V'; \xi (u') = u \wedge
  \hat \Delta (u', \edge{u}{v}) = (\mathbf{a}_{u,U} \mathbf{x}_u \leq b_{u,U}, C_u) \right) \right\rbrace
\\
\end{align*}

This intuitively means that constraints are duplicated according to
the shape of copy tree $T$.  Given the constraint for the root node in
the copy tree as
\[ \hat \Delta (v_{v_\bot}, \bot) = (\mathbf{a}_0 \mathbf{x}_{v_\bot} \leq b_0, C_{v_\bot}) \]
a model to $C_{v_\bot}$ gives a solution $\rho$ to $G$, if one exists.

\begin{align*}
 \rho = \left\lbrace
  \left( P, \bigwedge_U \sigma(\mathbf{a}_{v,U} \mathbf{x}_v \leq b_{v,U}) \right) \middle|
  \forall v \in V_T; \varphi(v) = P(\mathbf{x}_v)
 \right\rbrace
\end{align*}

If $C_{v_\bot}$ stays unsatisfiable, we let $T$ grow based on the
unsatisfiable core for $C_{v_\bot}$.

\subsubsection{Refinement from Unsatisfiable Core}

We first start the process from the initial Copy tree:
\begin{align*}
V_C = & \left\lbrace v_{v_\bot} \right\rbrace \\
E_C = & \emptyset
\end{align*}

An unsatisfiable core $\mathcal{U}$ is a subset of $C_{v_\bot}$.
If a vertex $v \in V$ with the copy origin $U$ satisfies:
\begin{align*}
& \exists t, u \in V; v \leadsto t \wedge v \leadsto u \wedge \\
& \left( \exists i, j, k;
\left\lbrace \left( \mathbf{a}_{s,U,j} = \ldots + \mathbf{a}_{v,U,i} + \ldots \right),
\left( \mathbf{a}_{t,U,k} = \ldots + \mathbf{a}_{v,U,i} + \ldots \right)
\right\rbrace \subseteq \mathcal{U} \right)
\end{align*}
we add $v$ to the copy tree $T$.

\subsection{Acyclic Horn problems}

We eliminate the restriction that all Horn term vertices
$V_\Rightarrow$ have exactly one successor.

We define selectors for preceding disjunction vertices to $v$.
\[ \mathcal{S}_v = \prod_{u \in \mathrm{succ}(v)} \left\lbrace u \mapsto t \middle| \edgel{t}{\theta}{u} \right\rbrace \]

\begin{align*}
K_l = \begin{cases}
\left\lbrace l \right\rbrace & \mbox{if } l = \mathbf{a} \mathbf{x} \leq b \\
\mathcal{X}_P \cup \left\lbrace \bot \right\rbrace
& \mbox{if } l = P(\mathbf{x})
\end{cases} \\
\forall s \in \mathcal{S}_v;
X = \prod_{t \in \mathrm{ran}(s)} K_{\varphi(t)}
\exists c \in X;
\end{align*}

We define a \textbf{selector set collection} \cite{albarghouthi13} for
every Horn term vertices, which is a set of sets of selectors.

For $v \in V_T$, the selector set collection $\mathcal{X}_v =
\left\lbrace S_{v,1}, \ldots, S_{v,\kappa_v} \right\rbrace$ consists
of $\kappa_v$ sets of non-empty and non-overlapping subsets of
$\mathcal{S}$. If $\nexists u; u \rightarrow v$, $\mathcal{X}_v =
\emptyset$.

We arbitrarily choose the initial selector $s_0 \in \mathcal{S}$. We
start from empty collections for all $v$.

\bibliographystyle{plain}
\bibliography{./biblio}

\end{document}
