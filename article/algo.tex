\documentclass[a4paper,12pt]{article}

\usepackage{amsmath}
\usepackage{amssymb}
\usepackage{algorithm}
\usepackage{algorithmic}
\usepackage{listings}
\usepackage{stmaryrd}
\usepackage{syntax}

\newcommand{\sembrack}[1]{[\![#1]\!}
\newcommand{\edge}[2]{#1\rightarrow#2}
\newcommand{\path}[2]{#1\rightarrow^\star#2}

\title{Algorithm explanation}
\author{Yuto Takei \\ The University of Tokyo}

\begin{document}
\maketitle

\section{Algorithm}

The Horn clause solving algorithm receives an input problem in Horn
graph. A Horn graph is a labelled directed graph $G=(V,E,\varphi)$.
\begin{itemize}
\item $V$ is a union of two sets of vertices; Horn term vertices $V_T$
  and arrow vertices $V_\Rightarrow$. Each vertex $u \in
  V_\Rightarrow$ have eactly one outgoing edge to some $v \in V_T$,
  denoted as $succ(u)$.
\item $E$ is a set of labelled edges, which are expressed in the form
  $(u,v,\theta)$ where $\theta$ is a finite map from variables to
  variables.
\item $\varphi: V_T \rightarrow L$ is a labelling function for Horn
  term vertices, where L is a set of elements of a predicate variable
  in the form $P(x_1, ..., x_{arity(P)})$ or a linear inequality $a_1
  x_1 + \cdots + a_n x_n \leq b$.
\end{itemize}

There are two kinds of edges in a Horn graph;
\begin{itemize}
\item for $u \in V_T, v \in V_\Rightarrow$, an edge takes a form
  $(u,v,\theta)$, or,
\item for $u \in V_\Rightarrow, v \in V_T$, an edge takes a form
  $(u,v,\emptyset)$, which the mapping is empty.
\end{itemize}

The meaning $\llbracket G \rrbracket $ of a Horn graph $G$ is a set of
Horn clauses $\mathcal{HC}_v$ for all vertices $v \in V_\Rightarrow$.

\begin{align*}
\mathcal{HC}_v & = \left( \bigwedge_{(u,v,\theta) \in E} \theta \varphi(u) \right) \Longrightarrow \varphi(succ(v)) \\
\llbracket G \rrbracket & = \bigwedge_{v \in V} \mathcal{HC}_v
\end{align*}

We implicitly assume each $\mathcal{HC}_v$ is universally quantified
by its all free variables $FV(\mathcal{HC}_v)$.

Note that we allow multiple appearance of Horn clauses with the same
predicate variable on their right-hand side.

The solution of $G$ is a map $\rho$ from predicate variables to linear
predicates of the form $\lambda x_1, \cdots ,x_n. \psi $ (where $\psi$
is a formula constructed from
\begin{align*}
\psi ::= & a_1 x_1 + \cdots + a_n x_n \leq b \mid \\
& \psi \wedge \psi \mid \psi \vee \psi \mid true \mid false
\end{align*}
), where $\rho[G]$ is a tautology. A solution is \textit{simple} if
$\rho[P]$ is of the form $a_1 x_1 + \cdots + a_n x_n \leq b$ for every
$P$ in $dom(\rho)$.

The problem of solving Horn clauses (represented by $G$) is to find a
solution of $G$ (and a simple solution if possible).

We consider below a Horn graph that is acyclic.

Given an acyclic Horn graph (or called a Horn DAG), we define the
relation $u <^+ v$ over $V_T$ as follows.
\[ u <^+ v \Longleftrightarrow \exists a, \theta; \left\lbrace{ (u,a,\theta), (a,v,\emptyset) \right\rbrace \subseteq E \]
By the assumption that $G$ is acyclic, $<^+$ is a well-founded
relation. We assume without loss of generality that $V_T$ contains the
greatest element $v_\bot$ with respect to $<^+$; otherwise we can
always implicitly assume $\varphi(v_\bot) = 0 \leq 0$ such that $\forall v \in V_\bot; \exists v_a; \left\lbrace (v, v_a, \emptyset), (v_a, v_\bot, \emptyset) \right\rbrace \subseteq E $.

For each $v \in V_T$, we define
$\Delta(v) = (\mathbf{a}_v \mathbf{x} \leq b_v, C_v)$
, where
\begin{itemize}
\item $\mathbf{a}_v \mathbf{x} \leq b_v$ is a linear inequality that
  may contain coefficient variables
\item $C_v$ is a set of linear expressions on coefficient variables.
\end{itemize}
Intuitively, the pair $(\mathbf{a} \mathbf{x} \leq b, C)$ denotes the
set of inequalities:

\begin{align*}
\left\lbrace
 \mathbf{a} \mathbf{x} \leq b \middle|
 \exists \left( FV(C)
  \setminus (\mathbf{a} \cup \left\lbrace b \right\rbrace
 \right); C
\right\rbrace
\end{align*}

$\Delta(v)$ is defined by well-founded induction on $v$ with respect
to the relation $<^+$. Then,
$\Delta(v) = \left( \mathbf{a}_v \mathbf{x} \leq b_v, C_v \right)$,
where

\begin{align*}
C_v = & \bigcup_{u <^+ v} C_u
\\
& \cup \begin{cases}
\emptyset
& \mbox{if } \nexists u; u <^+ v \\
\left\lbrace
 \mathbf{a}_v = \sum_{u <^+ v} \mathbf{a}_u,
 b_v \geq \sum_{u <^+ v} b_u
\right\rbrace
& \mbox{otherwise}
\end{cases}
\\
& \cup \begin{cases}
\emptyset
& \mbox{if } \varphi(v) = P(\mathbf{x}) \\
\left\lbrace
 \lambda_v \geq 0, \mathbf{a}_v = \lambda_v \mathbf{a}_0,
 b_v = \lambda_v b_0
\right\rbrace
& \mbox{if } \varphi(v) = \mathbf{a}_0 \mathbf{x} \leq b_0
\end{cases}
\end{align*}

Any model $\sigma$ to the constraint $C_{v_\bot}$ gives a simple
solution for $G$ if $C_{v_\bot}$ is satisfiable, i.e., we should have:

\begin{quote}
If $\models \sigma(C_{v_\bot})$, then $\rho$ defined by
\begin{align*}
 \rho = \left\lbrace
  \left( P, \sigma(\mathbf{a}_v \mathbf{x} \leq b_v) \right) \middle|
  \forall v \in V; \varphi(v) = P(\mathbf{x})
 \right\rbrace
\end{align*}
is a solution of $G$.
\end{quote}

If $C_{v_\bot}$ is unsatisfiable, there are two possibilities for them.

\begin{itemize}
\item No solution exists for $G$.
\item No simple solution exists for $G$.
\end{itemize}

If $G$ has a vertex which has multiple successors, the second case is
possible. We now try to give a solution by relaxing the constraint
$C_{v_\bot}$.

We slightly modify the definition $C_v$. Let $\hat V$ a subset of
vertices with multiple successors.

\begin{align*}
C_v = & \bigcup_{\substack{u <^+ v \\ u \not\in \hat V}} C_u \cup
\bigcup_{\substack{u <^+ v \\ u \in \hat V}} \textsc{Copy}_v (C_u)
\\
& \cup \begin{cases}
\emptyset
& \mbox{if } \nexists u; u <^+ v \\
\left\lbrace
 \mathbf{a}_{v,\emptyset} =
  \sum_{\substack{u <^+ v \\ u \not\in \hat V}} \mathbf{a}_{u,U} +
  \sum_{\substack{u <^+ v \\ u \in \hat V}} \mathbf{a}_{u,U \cup \left\lbrace v \right\rbrace},
 b_{v,\emptyset} \geq
  \sum_{\substack{u <^+ v \\ u \not\in \hat V}} b_{u,U} +
  \sum_{\substack{u <^+ v \\ u \in \hat V}} b_{u,U \cup \left\lbrace v \right\rbrace}
\right\rbrace
& \mbox{otherwise}
\end{cases}
\\
& \cup \begin{cases}
\emptyset
& \mbox{if } \varphi(v) = P(\mathbf{x}) \\
\left\lbrace
 \lambda_v \geq 0, \mathbf{a}_{v,\emptyset} = \lambda_v \mathbf{a}_0,
 b_{v,\emptyset} = \lambda_v b_0
\right\rbrace
& \mbox{if } \varphi(v) = \mathbf{a}_0 \mathbf{x} \leq b_0
\end{cases}
\end{align*}

The procedure $\textsc{Copy}_v$ renames all free variables to avoid
name collision. Its definition is given as follows.

\[\textsc{Copy}_{\hat v}(\mathbf{a}_{u,V}) = \mathbf{a}_{u,(V \cup
  \left\lbrace \hat v \right\rbrace)} \]

All variables are then distinguished by the label of copy origin
nodes. This intuitively means that constraints generated for nodes in
$\hat V$ are duplicated at their successors. Note that the variables
without the label should be interpreted as that it has $\emptyset$ for
origin nodes.

When $C_{v_\bot}$ becomes satisfiable by specifing an appropriate
$\hat V$, a solution $\rho$ to $G$ is given as follows.

\begin{align*}
 \rho = \left\lbrace
  \left( P, \bigwedge_S \sigma(\mathbf{a}_{v,S} \mathbf{x} \leq b_{v,S}) \right) \middle|
  \forall v \in V; \varphi(v) = P(\mathbf{x})
 \right\rbrace
\end{align*}

If $C_{v_\bot}$ stays unsatisfiable, we can add a vertex with multiple
successors to $\hat V$ until all such nodes are in $\hat V$.

\end{document}
