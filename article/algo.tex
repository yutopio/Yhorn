\documentclass[a4paper,12pt]{article}

\usepackage{amsmath}
\usepackage{amssymb}
\usepackage{algorithm}
\usepackage{algorithmic}
\usepackage{graphicx}
\usepackage{listings}
\usepackage{stmaryrd}
\usepackage{syntax}

\newcommand{\sembrack}[1]{[\![#1]\!}
\newcommand{\edge}[2]{#1\rightarrow#2}
\newcommand{\edgel}[3]{#1\xrightarrow{#2}#3}

\title{Algorithm explanation}
\author{Yuto Takei \\ The University of Tokyo}

\begin{document}
\maketitle

\section{Algorithm}

\subsection{Example}

\paragraph {Tree interpolation}
First, consider the following example.

\begin{align*}
x \leq 1 & \implies A(x) \\
y \leq 2 & \implies B(y) \\
A(x) \wedge B(y) & \implies 3x+2y \leq 10
\end{align*}

In this example, we would like to compute linear arithmetic
formulas for the predicate variables $A$ and $B$.
They appear only once on both left-hand and right-hand globally in the problem.

We try to obtain a simple solution that is expressed by linear expression
templates, whose coefficients and constants are parameterized. Assume
that they have following predicate templates.

\begin{align*}
A(x) : c_1 x + c_2 \leq 0 \\
B(y) : c_3 y + c_4 \leq 0
\end{align*}

By naturally encoding the first Horn clause with Farkas's lemma,
we obtain a linear constraint
$ \lambda_1 = c_1 \wedge - \lambda_1 \geq c_2 \wedge \lambda_1 \geq 0 $.
Any assignment to $ c_1, c_2, \lambda_1 $ which satisfies this will make $A(x)$
valid for the first clause. In the same manner, we obtain
$ \lambda_2 = c_3 \wedge - 2 \lambda_2 \geq c_4 \wedge \lambda_2 \geq 0 $
from the second clause, and
$ c_1 = 3 \lambda_3 \wedge c_3 = 2 \lambda_3 \wedge c_2 + c_4 \geq -10 \lambda_3 \wedge \lambda_3 > 0 $
from the third.

One of the models satisfying all the constraints above is
$( c_1, c_2, c_3, c_4, \lambda_1, \lambda_2, \lambda_3 ) = (3, -3, 2, -4, 3, 2, 1)$.
By assigning them to the template, we obtain the following solution.

\begin{align*}
A(x) : 3 x - 3 \leq 0 \\
B(y) : 2 y - 4 \leq 0
\end{align*}

\paragraph {DAG Horn graph solving}
Second, we explain a case that some predicate variables appear
multiple times on left-hand side in different Horn clauses.

\begin{align*}
x \leq 1 & \implies A(x) \\
A(x) \wedge A(y) & \implies x+y \leq 2
\end{align*}

Assigning a template
$A(x) : c_1 x + c_2 \leq 0$, we obtain the same
constraint for the first clause as the previous example.

\begin{align} \label{eq:constr1}
\lambda_1 = c_1 \wedge - \lambda_1 \geq c_2 \wedge \lambda_1 \geq 0
\end{align}

Sinve we have $A$'s appearance twice on the second clause,
it is possible for $A$ to have two expressions each of which is used
for $A(x)$ and $A(y)$. Although we need to use the same template for
multiple occurrences in order to compute the simplest solution,
it may make the whole constraints unsatisfiable if the simplest
solution does not exist.

Therefore, we generate a weaker constraint for the second clause that
allows different expressions of $A$ for multiple occurrence. If the
whole constraint is satisfiable in the case, we try to solve in a
stronger constraint to generate a simple solution.

In this specific case, it is virtually equivalent to solve the
following problem in a weaker constraint.

\begin{align*}
\text{Problem:} \\
x \leq 1 & \implies A_1(x) \\
x \leq 1 & \implies A_2(x) \\
A_1(x) \wedge A_2(y) & \implies x+y \leq 2 \\
\\
\text{Templates:} \\
A_1(x) \text{ : } & c_{1,1} x + c_{2,1} \leq 0 \\
A_2(x) \text{ : } & c_{1,2} x + c_{2,2} \leq 0 \\
A(x) \text{ : } & A_1(x) \wedge A_2(x) \\
\end{align*}

By duplicating the constraint~\ref{eq:constr1} of $A$ for newly
created $A_1$ and $A_2$, we obtain the whole constraint as follows.

\begin{align*}
& \lambda_{1,1} = c_{1,1} \wedge - \lambda_{1,1} \geq c_{2,1} \wedge \lambda_{1,1} \geq 0 \wedge \\
& \lambda_{1,2} = c_{1,2} \wedge - \lambda_{1,2} \geq c_{2,2} \wedge \lambda_{1,2} \geq 0 \wedge \\
& c_{1,1} = \lambda_2 \wedge c_{1,2} = \lambda_2 \wedge c_{2,1} + c_{2,2} \geq 2 \lambda_2 \wedge \lambda_2 > 0
\end{align*}

This constraint is satisfiable and one model is
\[ ( c_{1,1}, c_{1,2}, c_{2,1}, c_{2,2}, \lambda_{1,1}, \lambda_{1,2}, \lambda_2 ) =
( 1, -1, 1, -1, 1, 1, 1 ) \] and the solution then becomes
$A(x) : (x -1 \leq 0) \wedge (x -1 \leq 0)$.

Since the weak constraint is satisfiable, we now refine the constraint
to obtain the better solution. That is, simply eliminate the
duplication to build the following constraint for the original template.

\begin{align*}
& \lambda_1 = c_1 \wedge - \lambda_1 \geq c_2 \wedge \lambda_1 \geq 0 \wedge \\
& c_1 = \lambda_2 \wedge c_1 = \lambda_2 \wedge 2 c_2 \geq 2 \lambda_2 \wedge \lambda_2 > 0
\end{align*}

Solving this constraint simply gives us a solution $A(x) : x -1 \leq 0$.

\paragraph {Disjunctive Horn graph solving}
Finally, we explain a case that an input set of Horn clauses contain
disjunctions on left-hand side.

\begin{align*}
x \leq 0 \wedge -x \leq 0 & \implies A(x) \\
x+1 \leq 0 \vee -x+1 \leq 0 & \implies B(x) \\
A(x) \wedge B(x) & \implies 1 \leq 0
\end{align*}

Same as the previous examples, templates for predicate variables are
prepared.

\begin{align*}
A(x) : c_1 x + c_2 \leq 0 \\ B(x) : c_3 x + c_4 \leq 0
\end{align*}

The constraints generated are follows.

\begin{align*}
& \lambda_1 - \lambda_2 = c_1 \wedge 0 \geq c_2 \wedge \lambda_1 \geq 0 \wedge \lambda_2 \geq 0 \wedge \\
& \lambda_3 = c_3 \wedge \lambda_3 \geq c_4 \wedge \lambda_3 \geq 0 \wedge \\
& - \lambda_4 = c_3 \wedge \lambda_4 \geq c_4 \wedge \lambda_4 \geq 0 \wedge \\
& c_1 + c_3 = 0 \wedge c_2 + c_4 > 0
\end{align*}

The constraints here become unsatisfiable, and one possible
unsatisfiable core is:

\begin{align*}
\lambda_3 & = c_3 \\
\lambda_3 & \geq c_4 \\
- \lambda_4 & = c_3 \\
\lambda_4 & \geq c_4 \\
c_2 + c_4 & > 0 \\
0 & \geq c_2
\end{align*}

Because the unsatisfiable core contains constraints originating from
the disjunctions to be expressed in a single term template. This let
us determine to relax the template to allow a enlarged solution.

Now we slightly change the problem and the template.

\begin{align*}
\text{Problem:} \\
x \leq 0 \wedge -x \leq 0 & \implies A(x) \\
x+1 \leq 0 & \implies B_1(x) \\
-x+1 \leq 0 & \implies B_2(x) \\
A(x) \wedge B_1(x) & \implies 1 \leq 0 \\
A(x) \wedge B_2(x) & \implies 1 \leq 0 \\
\\
\text{Templates:} \\
A(x) \text{ : } & c_1 x + c_2 \leq 0 \\
B_i(x) \text{ : } & c_{3,i} x + c_{4,i} \leq 0 \quad (i \in \left\lbrace 1,2 \right\rbrace ) \\
B(x) \text{ : } & B_1(x) \vee B_2(x)
\end{align*}

When we construct the constraints, as same as previous, we again
obtain an unsatisfiable core:

\begin{align*}
\text{Constraint:} \\
TODO \\
\\
\text{Unsatisfiable core:} \\
TODO
\end{align*}

This time, the unsatisfiable core contains the constraints that
originate from

TODO

\begin{align*}
\text{Problem:} \\
x \leq 0 \wedge -x \leq 0 & \implies A_i(x) \quad (i \in \left\lbrace 1,2 \right\rbrace ) \\
x+1 \leq 0 & \implies B_1(x) \\
-x+1 \leq 0 & \implies B_2(x) \\
A_i(x) \wedge B_j(x) & \implies 1 \leq 0 \quad (i,j \in \left\lbrace 1,2 \right\rbrace ) \\
\\
\text{Templates:} \\
A_i(x) \text{ : } & c_{1,i} x + c_{2,i} \leq 0 \quad (i \in \left\lbrace 1,2 \right\rbrace ) \\
A(x) \text{ : } & A_1(x) \wedge A_2(x) \\
B_i(x) \text{ : } & c_{3,i} x + c_{4,i} \leq 0 \quad (i \in \left\lbrace 1,2 \right\rbrace ) \\
B(x) \text{ : } & B_1(x) \vee B_2(x)
\end{align*}

Finally, we obtain the following constraint:

TODO

\subsection{Preliminaries}

Our Horn clause solving algorithm receives an input problem in Horn
graph. A \textbf{Horn graph} is a labelled directed acyclic graph $G=(V,E,\varphi)$.
\begin{itemize}
\item $V$ is a union of two disjoint sets of vertices; Horn term
  vertices $V_T$ and arrow vertices $V_\Rightarrow$. Each vertex $u
  \in V_\Rightarrow$ has eactly one outgoing edge to some $v \in V_T$,
  denoted as $succ(u)$.
\item $E$ is a set of labelled edges, which are expressed in the form
  $(u,v,\theta)$ where $\theta$ is a finite map from variables to
  variables.
\item $\varphi: V_T \rightarrow L$ is a labelling function for Horn
  term vertices, where L is a set of elements of a predicate variable
  in the form $P(x_1, \ldots, x_{\mathrm{arity}(P)})$ or a linear
  inequality $a_1 x_1 + \cdots + a_n x_n \leq b$.
\end{itemize}
There are two kinds of edges in a Horn graph;
\begin{itemize}
\item for $u \in V_T, v \in V_\Rightarrow$, an edge takes a form
  $(u,v,\theta)$, written $\edgel{u}{\theta}{v}$, or,
\item for $u \in V_\Rightarrow, v \in V_T$, an edge takes a form
  $(u,v,\emptyset)$, written $\edge{u}{v}$, where the mapping is
  empty.
\end{itemize}

The meaning $\llbracket G \rrbracket $ of a Horn graph $G$ is a set of
Horn clauses $\mathcal{HC}_v$ for all vertices $v \in V_\Rightarrow$.

\begin{align*}
\mathcal{HC}_v & = \left( \bigwedge_{(u,v,\theta) \in E} \theta \varphi(u) \right) \Longrightarrow \varphi(succ(v)) \\
\llbracket G \rrbracket & = \bigwedge_{v \in V_\Rightarrow} \mathcal{HC}_v
\end{align*}

We implicitly assume each $\mathcal{HC}_v$ is universally quantified
by its all free variables $\textsc{FV}(\mathcal{HC}_v)$.  Without loss
of generality, we limit linear expression labels on $V_T$ vertices
with incoming edges to $1 \leq 0$, meaning $\bot$.

The solution of $G$ is a map $\rho$ from predicate variables to linear
predicates of the form $\lambda x_1, \cdots ,x_n. \psi $ (where $\psi$
is a formula constructed from
\begin{align*}
\psi ::= & a_1 x_1 + \cdots + a_n x_n + b \leq 0 \mid \\
& \psi \wedge \psi \mid \psi \vee \psi
\end{align*}
), where $\rho[G]$ is a tautology. A solution is \textit{simple} if
$\rho[P]$ is of the form $a_1 x_1 + \cdots + a_n x_n + b \leq 0$ for every
$P$ in $\mathrm{dom}(\rho)$.

The problem of solving Horn clauses (represented by $G$) is to find a
solution of $G$ (and a simple solution if possible).

\subsection{Simple solution}

We first propose an algorithm to compute a simple solution for the
given $G$ if one exists.  Later we add modifications to the algorithm
for the case that a simple solution does not exist.

Our algorithm works in a constructive manner.
Algorithm~\ref{alg:solveHorn} shows the pseudo-code of our algorithm.

\newcommand{\solveHorn}{\ensuremath{\mbox{\sc SolveHornGraph}}}
\newcommand{\genConstr}{\ensuremath{\mbox{\sc GenerateConstraint}}}
\newcommand{\splitGraph}{\ensuremath{\mbox{\sc RestructureGraph}}}

\begin{algorithm}
\caption{$ \solveHorn (G) $}\label{alg:solveHorn}
\begin{algorithmic}
\STATE {$v_\bot \gets $ the root node of $G$}
\STATE {$V \gets \left\lbrace v_\bot \right\rbrace$}
\REPEAT
  \STATE {$C \gets \genConstr (v_\bot, V)$}
  \STATE {$c_{v_\bot} \gets C(v_\bot)$}
  \STATE {Evaluate satisfiability of $c_{v_\bot}$}
  \IF {$c_{v_\bot}$ is unsatisfiable}
    \STATE {$\mathcal{UC} \gets $ unsatisfiable core for $c_{v_\bot}$}
    \STATE {$G \gets \splitGraph (G, \mathcal{UC})$}
    \STATE {$V \gets \left\lbrace v_\bot \right\rbrace$}
  \ELSE
    \STATE {Enlarge $V$}
    \STATE {$m \gets $ one model of $c_{v_\bot}$}
  \ENDIF
\UNTIL {$V$ contains all predicate variable vertices}
\STATE {Build solution by $m$}
\end{algorithmic}
\end{algorithm}


\begin{algorithm}
\caption{$ \genConstr (v, visited) $}\label{alg:genConstr}
\begin{algorithmic}
\STATE {$m \gets \emptyset$}
\FORALL {$v_\Rightarrow \in preds(v)$}
  \FORALL {$v_p \in preds(v_\Rightarrow)$}
    \STATE { $ l := \varphi (v_p) $ }
    \IF {$l$ is a predicate variable}
      \STATE {$m' \gets \genConstr (v_p) $}
      \STATE {$m \gets m \cup m'$}
    \ELSE
      \STATE { $id$ }
    \ENDIF
  \ENDFOR
\ENDFOR
\end{algorithmic}
\end{algorithm}

The construction of the algorithm is in two fold.
\begin{itemize}
\item \textbf{Iterative Constraint Building} The process starts from
  generating the weakest constraint that the graph allows and
  iteratively strengthen a constraint until the strongest one.
\item \textbf{Graph Restructuring} If the constraint become
  unsatisfiable during the iteration, restructure the input Horn graph
  to enlarge the solution template.
\end{itemize}

All vertices with predicate variable labels are assigned a simple
template.  For a vertex with a label $P(\mathbf{x})$, the algorithm
assigns a template $\mathbf{a_P} \mathbf{x} + b_P \leq 0$.  Parameters
$\mathbf{a_P}$ and $b_P$ are specified by a linear constraint.

For each predicate variable vertex, build a linear constraint
  that parameters in predicate variable templates must satisfy.



$\genConstr$ procedure produces the constraint for all vertices


In this algorithm, vertices with a predicate variable label must have
a simple solution assignment.



For simplicity of discussion, we define \textbf{term arity} $k_v$ for
every Horn term vertex $v \in V_T$ with incoming edges.
\begin{align*}
k_v =
\begin{cases}
\mathrm{arity}(P) & \mbox{if } \varphi(v) = P(x_1,...,x_{\mathrm{arity}(P)}) \\
0 & \mbox{if } \varphi(v) = \top \mbox { or } \varphi(v) = \bot
\end{cases}
\end{align*}
We let $K$ denote the maximum of $k_v$ over all $v \in V_T$.

We define the relation $\leadsto$ over $V_T$ as follows.
\[ u \mathop{\leadsto}^\theta v \Longleftrightarrow
\exists a \in V_\Rightarrow; \edgel{v}{\theta}{a} \wedge \edge{a}{v} \]
We may omit $\theta$ and write $u \leadsto v$ for convenience in later
discussions.

By the assumption that $G$ is acyclic, $\leadsto^+$ is a well-founded
relation. We assume without loss of generality that $V_T$ contains the
greatest element $v_\bot$ with respect to $\leadsto^+$; otherwise we
can always implicitly assume $\varphi(v_\bot) = \top$ such that
$\forall v \in V_T \setminus \{v_\bot\}; v \mathop{\leadsto}^\emptyset v_\bot$.

For each $v \in V_T$, we define
$\Delta(v) = (\mathbf{a}_v \mathbf{x}_v \leq b_v, C_v)$
, where
\begin{itemize}
\item $\mathbf{a}_v \mathbf{x}_v \leq b_v$ is a linear inequality that
  contains coefficient variables $\mathbf{a}_v$, and
\item $C_v$ is a set of linear expressions on coefficient variables.
\end{itemize}
Intuitively, the pair $(\mathbf{a} \mathbf{x} \leq b, C)$ denotes the
set of inequalities:
\begin{align*}
\left\lbrace
 \mathbf{a} \mathbf{x} \leq b \middle|
 \exists \left( \textsc{FV}(C)
  \setminus (\mathbf{a} \cup \left\lbrace b \right\rbrace
 \right); C
\right\rbrace
\end{align*}

$\Delta(v)$ is defined by well-founded induction on $v$ with respect
to the relation $\leadsto^+$. Then, $C_v$ in
$\Delta(v) = \left( \mathbf{a}_v \mathbf{x}_v \leq b_v, C_v \right)$
can be shown as follows.

\begin{align*}
\hat C_{a,v} = &
 \bigcup_{\edgel{u}{\theta}{a}} C_u \cup
 \bigcup_{0 < i \leq K}
 \left\lbrace
  \mathbf{a}_{v,i} = \sum_{\edgel{u}{\theta}{a}} \mathbf{a}_{u, \theta^{-1} (i)}
 \right\rbrace \\
 & \hspace{2cm} \cup
 \left\lbrace
  b_v \geq \sum_{\edgel{u}{\theta}{a}} b_u
 \right\rbrace \cup
 \bigcup_{k_v < i \leq K}
 \left\lbrace \mathbf{a}_{v,i} = 0 \right\rbrace
\\
C_v = & \bigcup_{\edge{a}{v}} \hat C_{a,v} \cup
\begin{cases}
\emptyset
& \mbox{if } \varphi(v) = P(\mathbf{x}_v) \\
\left\lbrace
 \lambda_v \geq 0, \mathbf{a}_v = \lambda_v \mathbf{a}_0,
 b_v = \lambda_v b_0
\right\rbrace
& \mbox{if } \varphi(v) = \mathbf{a}_0 \mathbf{x}_v \leq b_0
\end{cases}
\end{align*}

For simplicity, we assume variables $\mathbf{a}_u$, $b_u$ are obtained
from $\Delta(u)$.

Any model $\sigma$ to the constraint $C_{v_\bot}$ gives a simple
solution for $G$ if $C_{v_\bot}$ is satisfiable, i.e., we should have:

\begin{quote}
If $\models \sigma(C_{v_\bot})$, then $\rho$ defined by
\begin{align*}
 \rho = \left\lbrace
  \left( P, \sigma(\mathbf{a}_v \mathbf{x}_v \leq b_v) \right) \middle|
  \forall v \in V_T; \varphi(v) = P(\mathbf{x}_v)
 \right\rbrace
\end{align*}
is a solution of $G$.
\end{quote}

\subsection{Disjunction-free problems}

If $C_{v_\bot}$ is unsatisfiable, no simple solution exists for
$G$. We now try to give a solution by relaxing the constraint
$C_{v_\bot}$.

We consider below a Horn graph whose Horn term vertices has at
most one successor.

\subsubsection{Quantifier elimination}

A set of eliminating quantifiers are obtained as $quant(n)$.

\begin{align*}
in(n) & = \bigcup_{p\in pred(n)} out(p) \\
out(n) & = gen(n) \cup in(n) \\
dup(n) & = \left\lbrace t \middle| p,q \in pred(n); p \ne q \wedge
 t \in out(p) \wedge t \in out(q) \right\rbrace \\
quant(n) & = gen(n) \cup dup(n) \cup
 (out(n) \cap \bigcup_{s \in succ(n)} quant(s))
\end{align*}

Let $\hat V_T$ a set of Horn term vertices $V_T$ with multiple
successors.
\[ \hat V_T = \left\lbrace v \in V_T \middle|
\exists u, u', \theta, \theta';
\edgel{v}{\theta}{u} \wedge \edgel{v}{\theta'}{u'} \wedge
(u \ne u' \vee \theta \ne \theta') \right\rbrace \]
We define \textbf{Copy tree}, which is a tree $T=(V_C,E_C,\xi)$
rooted at $v_{v_\bot}$.
\begin{itemize}
\item $V_C$ is a set of nodes.
\item $E_C$ is a set of labelled edges, which are expressed in the
  form $(u,v,e)$, where $e \in E \cup {\bot}$.
\item $\xi: V_C \rightarrow \hat V_T \cup \left\lbrace v_\bot
  \right\rbrace$ is a labelling function for nodes. The root node
  $v_{v_\bot} \in V_C$ has a label $v_\bot$.
\end{itemize}

We provide $\textsc{Copy}_v$ procedure to rename all free variables
for avoiding name collision.
\[ \textsc{Copy}_v (x_V) = x_{V \cup \left\lbrace v \right\rbrace} \]
All variables with the same name are then distinguished by the label
of copy origin nodes $V$. For consistency, the variables without the
label should be interpreted as that it has $\emptyset$ for origin
nodes.

We give a constraint function over a copy tree
$\hat \Delta(v,e)$ for $v \in V_C, e \in E$ as
\[ \hat \Delta (v,e) = \textsc{Copy}_v (\Delta_{V'} (\xi (v)))
\text{ such that } V' = \left\lbrace v' \middle| (v, v',e) \in E_C \right\rbrace \]
extending $\Delta$ function to
$\Delta_{V'} (v) = (\mathbf{a}_{v,\emptyset} \mathbf{x}_v \leq b_{v,\emptyset}, C_v)$
as

\begin{align*}
\hat C_a = &
 \bigcup_{\edgel{u}{\theta}{a}} C_u \cup
 \bigcup_{0 < i \leq K}
 \left\lbrace
  \mathbf{a}_{v,i} = \sum_{\edgel{u}{\theta}{a}} \mathbf{a}_{u, \theta^{-1} (i)}
 \right\rbrace \\
 & \hspace{2cm} \cup
 \left\lbrace
  b_v \geq \sum_{\edgel{u}{\theta}{a}} b_u
 \right\rbrace \cup
 \bigcup_{k_v < i \leq K}
 \left\lbrace \mathbf{a}_{v,i} = 0 \right\rbrace
\\
C_v = & \bigcup_{\edge{a}{v}} \hat C_a \cup
\begin{cases}
\emptyset
& \mbox{if } \varphi(v) = P(\mathbf{x}_v) \\
\left\lbrace
 \lambda_v \geq 0, \mathbf{a}_{v,\emptyset} = \lambda_v \mathbf{a}_0,
 b_{v,\emptyset} = \lambda_v b_0
\right\rbrace
& \mbox{if } \varphi(v) = \mathbf{a}_0 \mathbf{x}_v \leq b_0
\end{cases}
\end{align*}

---

\begin{align*}
C_v = & \bigcup_{(u,U,\theta,C_u) \in S_v^-} C_u \cup
\bigcup_{(u,U,\theta,C_u) \in S_v^+} C_u
\\
& \cup
\left\lbrace
 \mathbf{a}_{v,\emptyset,i} =
  \sum_{(u,U,\theta,C_u) \in S_v^-} \mathbf{a}_{u,U,\theta^{-1}(i)} +
  \sum_{(u,U,\theta,C_u) \in S_v^+} \mathbf{a}_{u,U, \theta^{-1}(i)}
\right\rbrace \\
& \cup \left\lbrace
 b_{v,\emptyset} \geq
  \sum_{(u,U,\theta,C_u) \in S_v^-} b_{u,U} +
  \sum_{(u,U,\theta,C_u) \in S_v^+} b_{u,U}
\right\rbrace \\
& \cup \bigcup_{k_v < i \leq K} \left\lbrace \mathbf{a}_{v,\emptyset,i} = 0 \right\rbrace
\end{align*}
such that
\begin{align*}
& S_v^- = \left\lbrace (u, U, \theta, C_u) \middle|
  u \mathop{\leadsto}^\theta v \wedge
  \left( \nexists u' \in V'; \xi (u') = u \right) \wedge
  \Delta_{V'} (u) = (\mathbf{a}_{u,U} \mathbf{x}_u \leq b_{u,U}, C_u) \right\rbrace
\\
& S_v^+ = \left\lbrace (u, U, \theta, C_u) \middle|
  u \mathop{\leadsto}^\theta v \wedge
  \left( \exists u' \in V'; \xi (u') = u \wedge
  \hat \Delta (u', \edge{u}{v}) = (\mathbf{a}_{u,U} \mathbf{x}_u \leq b_{u,U}, C_u) \right) \right\rbrace
\\
\end{align*}

This intuitively means that constraints are duplicated according to
the shape of copy tree $T$.  Given the constraint for the root node in
the copy tree as
\[ \hat \Delta (v_{v_\bot}, \bot) = (\mathbf{a}_0 \mathbf{x}_{v_\bot} \leq b_0, C_{v_\bot}) \]
a model to $C_{v_\bot}$ gives a solution $\rho$ to $G$, if one exists.

\begin{align*}
 \rho = \left\lbrace
  \left( P, \bigwedge_U \sigma(\mathbf{a}_{v,U} \mathbf{x}_v \leq b_{v,U}) \right) \middle|
  \forall v \in V_T; \varphi(v) = P(\mathbf{x}_v)
 \right\rbrace
\end{align*}

If $C_{v_\bot}$ stays unsatisfiable, we let $T$ grow based on the
unsatisfiable core for $C_{v_\bot}$.

\subsubsection{Refinement from Unsatisfiable Core}

We first start the process from the initial Copy tree:
\begin{align*}
V_C = & \left\lbrace v_{v_\bot} \right\rbrace \\
E_C = & \emptyset
\end{align*}

An unsatisfiable core $\mathcal{U}$ is a subset of $C_{v_\bot}$.
If a vertex $v \in V$ with the copy origin $U$ satisfies:
\begin{align*}
& \exists t, u \in V; v \leadsto t \wedge v \leadsto u \wedge \\
& \left( \exists i, j, k;
\left\lbrace \left( \mathbf{a}_{s,U,j} = \ldots + \mathbf{a}_{v,U,i} + \ldots \right),
\left( \mathbf{a}_{t,U,k} = \ldots + \mathbf{a}_{v,U,i} + \ldots \right)
\right\rbrace \subseteq \mathcal{U} \right)
\end{align*}
we add $v$ to the copy tree $T$.

\subsection{Acyclic Horn problems}

We eliminate the restriction that all Horn term vertices
$V_\Rightarrow$ have exactly one successor.

We define selectors for preceding disjunction vertices to $v$.
\[ \mathcal{S}_v = \prod_{u \in \mathrm{succ}(v)} \left\lbrace u \mapsto t \middle| \edgel{t}{\theta}{u} \right\rbrace \]

\begin{align*}
K_l = \begin{cases}
\left\lbrace l \right\rbrace & \mbox{if } l = \mathbf{a} \mathbf{x} \leq b \\
\mathcal{X}_P \cup \left\lbrace \bot \right\rbrace
& \mbox{if } l = P(\mathbf{x})
\end{cases} \\
\forall s \in \mathcal{S}_v;
X = \prod_{t \in \mathrm{ran}(s)} K_{\varphi(t)}
\exists c \in X;
\end{align*}

We define a \textbf{selector set collection} \cite{albarghouthi13} for
every Horn term vertices, which is a set of sets of selectors.

For $v \in V_T$, the selector set collection $\mathcal{X}_v =
\left\lbrace S_{v,1}, \ldots, S_{v,\kappa_v} \right\rbrace$ consists
of $\kappa_v$ sets of non-empty and non-overlapping subsets of
$\mathcal{S}$. If $\nexists u; u \rightarrow v$, $\mathcal{X}_v =
\emptyset$.

We arbitrarily choose the initial selector $s_0 \in \mathcal{S}$. We
start from empty collections for all $v$.

\bibliographystyle{plain}
\bibliography{./biblio}

\end{document}
