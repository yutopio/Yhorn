\chapter{Experiment}
\label{chap:experiment}

We have incorporated our algorithm to the software model checker for
higher-order programs (MoCHi) \cite{conf/pldi/KobayashiSU11}.  It is
specifically designed to verify the correctness of higher-orde
programs written in a small subset of ML.  This verification tool
abstracts a given input program by inferring dependent types for
higher-order programs \cite{conf/ppdp/UnnoK09}.

We implemented our proposed algorithm into MoCHi by two steps.  First,
we replace an existing theorem prover with our interpolating theorem
prover as a sub-routine for solving Horn clauses in dependend typing.
We used a method to obtain a common small interpolant across different
interpolating problems, while reducing a Horn clause problem into
interpolation \cite{conf/ppdp/UnnoK09}.  Second, we completely
replaced MoCHi's Horn solving algorithm with our algorithm.

For simplicity of the implementation, we assume that the linear
inequalities in the integer space.  Because of this, our algorithm
lost the completeness.  By using Motzkin's transpose theorem
\cite{journals/networks/Rajan90}, it is possible to handle linear
inequalities in both $\mathbf{ax} + b \leq 0$ and $\mathbf{ax} + b <
0$.

While solving interpolating problems, our algorithm treated linear
equalities in the form $\mathbf{ax} + b = 0$ specially without
normalizing into a conjunction of linear equalities so that the
algorithm obtain a linear equality in a set of interpolants.  It is
because we observed some cases that requires a equality relation
between two variables but not an inequality as an program invariant.

Note that, however, in Horn clauses solving problems, when a predicate
variable $P$ needs to take an equality $=$ or negation $\neq$, the
predicate template grows during the solving process as
$\mathbf{ax}-b \leq 0 \wedge -\mathbf{ax}+b \leq 0$ in the former
case, and $\mathbf{ax}-b+1 \leq 0 \vee -\mathbf{ax}+b+1 \leq 0$ in the
latter case.  We also thought it possible to obtain predicates that
are similar to program invariants by considering multiple paths.

For linear constraint satisfaction, we used GNU Linear Programming Kit
(GLPK) first for its light computation load.  For Horn clause solving
however, Our implementation make use of Z3 \cite{conf/tacas/MouraB08}
as an SMT solver to obtain unsatisfiable core of the constraints.

\section{Interpolating theorem prover}

There are some interpolating theorem provers known.
\begin{itemize}
\item \textsc{CSIsat} \cite{conf/cav/BeyerZM08} combines linear
  arithmetic reasonings and SAT-based logical reasonings.
\item \textsc{FOCI} \cite{website/foci} extracts an interpolant by the
  proof-based method.
\item \textsc{CLP-Prover} \cite{website/clp} generates an interpolant
 from constraints by linear arithmetic reasonings. It only supports
 conjunctive problems.
\end{itemize}

In our experiment, we compared the performance of our algorithm with
CSIsat, that is mainly used as a subprocedure of MoCHi.  Note that
CSIsat may return different interpolants for an interpolating problem
because of the randomized algorithm.
