\documentclass{llncs}

\usepackage{fullpage,url}
\usepackage{algorithm}
\usepackage{algorithmic}
\usepackage{listings}
\usepackage{syntax}

\title{How to make a delicious pizza}
\author{Yuto Takei}
\institute{The University of Tokyo}

\begin{document}

\maketitle

\begin{abstract}
  This is the abstract.
\end{abstract}

\section{Introduction}

* Formal verification
--> Dependent typing of SHP
--> Horn clauses

* Some
Background for Horn clause solving
* Dependent type determination

Definition of Horn clause solving (Informal)

Advantages
* Solution space consideration


Discussion on Real space


\section{Algorithm Overview}

\subsection{Preliminaries}

\setlength{\grammarparsep}{-2pt}
\setlength{\grammarindent}{6em}
\begin{grammar}

<num> ::= All integers

<var> ::= Arithmetic variables

<term> ::= $<num> \cdot <var>$

<expr> ::=  $<term> + <term> + \cdots + <num> \leq 0$

<la> ::= $<expr>$ | $<la> \wedge <la>$

<pvar> ::= Predicate variables

<pred> ::= $<pvar> ( <var>, <var>, ... )$

<hTerm> ::= <la> | <pred>

<horn> ::= $<hTerm> \wedge <hTerm> \wedge \cdots \longrightarrow <hTerm>$

\end{grammar}



* We consider inequalities.
** NOTE: Even we have =, <>, >, <=, we can normalize to >=
         over integer space



\subsection{Generating a tree}

By receiving Horn clauses as an input for the algorithm, we build a
implication tree.

Def <Resolution Tree>:

Implication tree is such a tree that every node is labeled by logical
atoms. The label of every node is implied by the conjunction of its
children's label.


* Renaming?

\subsection{Application of Farkas' Lemma}

Correct statement of Farkas' Lemma
Relationship to Craig's Interpolation.

\subsection{Solution space representation}


\section{Extensions}

\subsection{Extension to DAG structure}

* Building graph

* Splitting a graph into tree

* Merging with the same-name predicate conjunctively

\subsection{Linear arithmetics with disjunction}

In the previous section, we assumed that the input inequalities are
conjunctively connected.

* Why do we need this extension? *

We extend our inequalities to include disjunctions in the linear
equations.

\setlength{\grammarindent}{2em}
\begin{grammar}
<la> ::= <expr> | $<la> \wedge <la>$ | $<la> \vee <la>$
\end{grammar}

* Predicates on the direct path from the leaf with disjunction to the
root are connected by disjunction
* Other predicates are connected by conjunction


* Why do we process LA-disjunction before DAG?
** DAG to LA-Disjunction are invalid???


\subsection{Merging of pred-pred}

* Simplification between clauses


\subsection{General Horn clauses with Loops}

No limitation Horn clause solving

* Merging pred-pred between looped same name


\section{Completeness and Complexity}


\section{Experiment}

We limited to integer space

\section{Future Work}
\section{Conclusion}

\bibliographystyle{plain}
\bibliography{./biblio}

\end{document}
