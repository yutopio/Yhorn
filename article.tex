\documentclass{llncs}

\usepackage{algorithm}
\usepackage{algorithmic}
\usepackage{listings}
\usepackage{syntax}

\title{Solving Horn clauses over LI}
\author{Yuto Takei}
\institute{The University of Tokyo}

\begin{document}
\maketitle
\begin{abstract}
  This is the abstract.
\end{abstract}

\section{Introduction}


* Formal verification
--> Dependent typing of SHP
--> Horn clauses

* Some
Background for Horn clause solving
* Dependent type determination

Horn clause solving is a problem of constraint solving that is consisted of multiple Horn clauses with unknown predicate symbols. The solution is an assignment for such symbols so that all given Horn clauses stay valid. There may be multiple solutions for one problem, and the algorithm is expected to show at least one.

In this paper, we present a novel algorithm to solve Horn clauses. As the unique feature of the algorithm, we preserve the solution space throughout the computation. We are able to provide multiple assignments for predicate symbols. Additionally, this technique enables us to handle general Horn clauses rather than the recursion-free ones, which was the major restriction of the previous work.

The rest of our paper is structured as follows: Section 2 shows the basic algorithm to solve recursion-free Horn clauses with solution space consideration. Section 3 extends this algorithm so that it works on general settings. Section 4 discusses the complexity and the completeness of the extended algorithm. Section 5 illustrates the experimental result. Finally Section 6 brings the conclusion.

\section{Algorithm}

In this section, we present a Horn clause solving algorithm with the conventional restriction, i.e., the input Horn clauses are recursion-less. However unlike the existing methods, our algorithm returns the set of solutions against a problem.

<The problem example>

For example, the above program input has at least two solutions as follows (but not limited to these).

<Solution 1>     <Soltuion 2>

Our method yields a set of solutions which includes two above. It is consisted of following major steps.

\begin{enumerate}
\item Creation of a implication tree out of input Horn clauses
\item Generation of linear constraints under consideration of all input linear inequalities
\item Output of a solution space with an assignment for all predicate symbols
\end{enumerate}

We give the definition of our problem and input/output of our algorithm with its syntax and semantics.

\subsection{Preliminaries}

In our problem setting, we are given a set of Horn clauses over linear arithmetic. They are consisted of linear inequalities on real space and unknown predicate symbols with parameters. Given real numbers $n$ and the binary relationship $leq$, the syntax is defined as follows. Let $v$ denote the real variables and $P$ denote the unknown predicate variables.

\setlength{\grammarindent}{6em}
\begin{grammar}

<t> ::= $n$ | $nv$ | $t + t$

<e> ::= $t \leq 0$ | $e \wedge e$

<p> ::= $P (v, v, ... )$

<x> ::= $e$ | $p$  % Horn term

<h> ::= $x \wedge x \wedge \cdots \longrightarrow x$

\end{grammar}


For the above syntax, we give the following interpretations.

<semantics definition>
[0=0] Top
[0=1] Bot

We further impose following constraints on our input.

\begin{enumerate}
\item The input does not have a Horn clause with only linear inequalities on both left-hand and right-hand of the clause. Even if there exists one, it does not affect the satisfiability of other clauses regardless of a predicate assignment if it is valid, and the whole problem does not have a solution otherwise. Thus, this constraint does not affect the generality of the problem.

\item (Relaxed in section 3.1) Each predicate symbols appears exactly once in both left-hand and right-hand on different Horn clauses (recursion-less Horn clauses).

\item For the convenience of the algorithm explanation, exactly one Horn clause has linear inequality on its right-hand. As the background of this constraint, we can split a problem with more than one such inequality into two smaller problems without a loss of generality by the previous constraint.

\item (Relaxed in section 3.4) When considering the relationship among Horn clauses by implications based on predicate symbols, it must not form a loop.

(TODO: We want to customize the bullet head to something like (A-1)...(A-4) for future reference.)

\end{enumerate}

Now we define the solution format for the problem. A solution to Horn clause solving problem is expressed by a set of assignment to unknown predicate symbols. Each assignment substitutes a predicate symbol with a linear equation. It must only includes variables which are bound by the parameter list of the predicate symbol. Given the set of assignments as a solution, all input Horn clauses must be valid after substitution.

<definition of solution>

Our algorithm receives a problem as defined above and, unlike existing algorithms, yields a pair of linear constraints on parameter variables and a set of assignments of parameterized linear expressions.

<output specification>
$$x := linear constraint on parameters
\theta := { [parameterized expressions on P / P] | P }
out := < x, \theta >
[ \theta P ] = [ parameterized expressions ]$$

In order to generate a solution to a problem out of our algorithm output, it is required to somehow find a set of values that satisfy linear constraints on parameters, and instantiate the parameterized linear expressions in the assignments.

\subsection{Generation of Implication Tree}

Our algorithm first generates an implication tree from the input Horn clauses.

\paragraph{Definition<Implication tree>} An implication tree is such a tree that all nodes are labeled by a Horn term. As the semantics of the tree, a Horn term on each node is implied by the conjunction of Horn terms on its children nodes.

In this paper, we may denote by a term "Horn term" a node itself in a implication tree unless it becomes ambiguous.

From a given Horn clause, we create a implication tree by letting the right-hand Horn term  be a parent node and each left-hand terms be children nodes. Because all predicate symbols appears exactly once on both left-hand and right-hand side of different clauses by the given restriction, it is possible to graft two different implication trees T1 and T2 at the predicate symbol P, given T1 which is generated from the Horn clause with P on its left-hand, and T2 equally that P on right-hand. Renaming on variables should accordingly take place when grafting trees. We show the pseudo-code for tree generating algorithm.

\begin{algorithm}
\caption{$\mbox{\sc BuildTree}$}
\begin{algorithmic}
\item pseudo-code
\end{algorithmic}
\end{algorithm}

The following statement is true about a generated implication tree.

\paragraph{Characteristics} Let $T$ denote an implication tree generated from a problem expressed by the set of Horn clauses $HC$. Every Horn term $x$ in $T$ that has children $Y$ corresponds to exactly one Horn clause Y |- x in HC. In the other way, a Horn clause (a, b, c |- z) in HC corresponds to exactly one sub-tree in T that consisted of node z with its children a, b, c.

All leaves and the root in the implication tree have a linear expression $e$, while others have a predicate variable $P$. Let $e_star$ denote the root. The following statement must be satisfied for a predicate symbol $P$, given $e_1,cdots,e_m$ as $P$'s descendants and $e_(m+1),cdots,e_n$ as other leaves.

$$E1,\cdots,em |- P. P \wedge (e(m+1), \cdots en) |- e_\star$$    	(1)

(TODO: How do we add expression numberings?)

For the later convenience, we represent $\neg e_\star$ as $e_(n+1)$.


\subsection{Craig's Interpolation}

Given two inconsistent prepositions A and B, we call a new preposition I a Craig's interpolant of A and B such that $I$ is implied by $A$, $I$ and $B$ are inconsistent, and $I$ includes only free variables that are used in both $A$ and $B$. The preposition P in \ref{expr:1} is a Craig's interpolant of $e_1,\cdots,e_m$ and $e_(m+1),\cdots,e_n,e_(n+1)$ by using the fact $e_1,\cdots,e_n,e_(n+1) |- bot$. We compute such interpolant by Farkas' Lemma.

\paragraph{Farkas' Lemma on linear inequalities} Let a linear inequality $e_i$ be represented as $a_i1 x_1 + \cdots + a_im x_m <= a_i0$. Assuming that $e_1,\cdots,e_n$ implies $e_0$, there exists $\lambda_1,\cdots,\lambda_n$ that satisfy $a_0j =\sum_(i=1)^n \lambda_i * a_ij (j=0...m)$.

(TODO: what should I do with the operator for $e_(n+1)$ ? It is a negated expression.)

Some variables in some inequalities may not exists in other inequalities. In such a case, we consider that the variables has a coefficient of $0$. According to the lemmma above, $\lambda_i$ exists with satisfying the constraint below when considering linear inequalities $e_1,\cdots,e_m$ and $e_(m+1),\cdots,e_n,e_(n+1)$.

<expr: Coeff's constraint
Const's constraint>

The interpolant I is then written as $c_1 x_1 + \cdots + c_m x_m <= c_0$ with parameterized coefficients $c_j$ satisfying $c_j =\sum_(i=1)^m \lambda _i * a_ij (j=0...m)$. For intuitive understanding, it is the linear combination of $e_1,...,e_m$ by the weight $\lambda _1,...\lambda _m$. Note that, with any assignments for $\lambda _i$ under the constraint, I is a correct interpolant.

\subsection{Solution Space Generation}

Let the predicate symbol $P$ have linear inequalities $e_1,...,e_m$ as its decendants in the implication tree. We denote all other leaves by $e_(m+1),...,e_n$. Our algorithm assigns to $P$ an interpolant between $e_1,...,e_m$ and $e_(m+1),...,e_n,e_(n+1)$. We hereby note that the constraint for $\lambda_i$ (see \ref{expr:2}) to obtain such interpolant. That constraint is common among computing the assignments to all predicate symbols. Thus we return the parameterized assignments and the constraint as the algorithm's output.

As we described in \ref{subsection:Preliminaries}, the solution for the Horn clause solving problem is the instanciation of all parameterized substitutions by solving the linear constraints of parameters. The correctness of the output as the problem solution can be verified by the induction on implication trees from characteristics \ref{}. We let $<a_1,...,a_(n+1)>$ represent one solution of $\lambda_i$ after solving the constraint. Indices in the statements below can be changed without loosing generality.

\begin{enumerate}
\item We consider the predicate symbol node P with only linear inequality nodes as children $e_1,...,e_m$. The algorithm assigns an interpolant between $e_1,...,e_m$ and all other inequalities $e_(m+1),...,e_(n+1)$ to $P$. Therefore $\theta P$ is a linear combination among $e_1,...,e_m$ with the weights $a_1,...,a_m$, and thus implied by the conjunction of $e_1,...,e_m$. The Horn clause implying P is valid.

\item We consider other predicate symbold P, having predicate symbols $Q_1,...,Q_k$ and linear expressions $e_1,...,e_m$ (if any) as its children. Assume that Horn clauses implying $Q_1,...,Q_k$ respectively are all valid. When we denote $Q_i$'s linear inequality decendants as $e_(m+t_1+\cdots+t_(i-1)+1),...,e_(m+t_1+\cdots+t_i)$, $P$'s assignment by the algorithm is an interpolant between $e_1,...,e_m+a_1+\cdots+a_k$ and $e_m+a_1+\cdots+a_k,\cdots,e_(n+1)$. While this interpolant is a linear combination among $e_1,...,e_m+a_1+\cdots+a_k$ by weights $a_1,...,a_m+a_1+\cdots+a_k$, when we note $\theta Q_i$ is a linear combination of $e_(m+t_1+\cdots+t_(i-1)+1),...,e_(m+t_1+\cdots+t_i)$ by weight $a_(m+t_1+\cdots+t_(i-1)+1),...,a_(m+t_1+\cdots+t_i)$, then $\theta P$ is clearly a linear combination among $e_1,...,e_m$ by weights $a_1,...,a_m$ and $\theta Q_1,...,\theta Q_k$ by weight all $1$. Therefore, the Horn clause implying P is valid.

\item We consider the root node $e_star$, having predicate symbols $Q_1,...,Q_k$ (if any) and linear expressions $e_1,...,e_m$ (if any) as its children. Let $e_(m+1),...,e_n$ be the all decendants of $Q_1,...,Q_k$. We can obtain the following expression from the $a_i$'s linear constraint.
$$ \theta Q_1 \wedge \cdots \wedge \theta Q_k \wedge e_1 \wedge \cdots e_m \wedge \neg e_\star |- \bot$$
By moving the $e_\star$ to the right-hand yields the Horn clause that we want.
\end{enumerate}


Therefore, we can say that the our alogrithm returns a valid solution space. As for the example shown in the beginning of this section, our algorithm returns the following linear constraint and parameterized assignments.

<algo output>

By solving a linear constraint, we may obtain following solutions for example.

(a1,....) = (~~~)
P: foo
Q: bar

\section{Applications}

\subsection{Supporting DAG-structured Horn clauses}

In the previous section, we imposed a restriction that each predicate symbol appears exactly once on both sides of different Horn clauses without making a looped dependency (A-2). Now we relax this by allowing the multiple appearance of predicate symbols without a looped dependency. Accompanied by this change, we relax (A-3) with consideration of multiple Horn clauses that the same predicate symbols implies different linear inequalities.

With the above relaxations, we perform the following changes to our algorithms.

\begin{itemize}
\item After receiving the input, we build an implication graph rather than an implication tree.
\item From the inequality node without outgoing edges, we traverse the implication graph to make implication trees. If we encounter the same predicate symbols multiple times during the traversal, we duplicate treat the symbol as a different one with renaming.
\item Finally we unify the duplicated symbols into one.
\end{itemize}

Firstly, we define an implication graph as follows.

\paragraph{Definition<Implication graph>} An implication graph is a directed multigraph that all nodes are labeled by a Horn term and all edges are labeled by a set of variable renaming. Semantically, a Horn term on each node is implied by Horn terms on its predecessor nodes with all variables on each node renamed by the edge's renaming rules.

The modified algorithm in this section generates an implication graph from input Horn clauses.

<pseudo-code>

The following statement is true about a generated graph.

\paragraph{Characteristics} Let $G$ denote an implication graph generated from a problem expressed by the set of Horn clauses $HC$. Every Horn term $x$ in $G$ that has children $Y$ corresponds to exactly one Horn clause Y |- x in HC. In the other way, a Horn clause (a, b, c |- z) in HC corresponds to exactly one sub-graph in $G$ that consisted of node z with its children a, b, c. (TODO: Do we need this description? Do we use this characteristics later?)

\paragraph{Characteristics} All the linear inequality nodes have degree 1. That is, each node has exactly one outgoing edge to a predicate symbol node, or otherwise exactly one incoming edge from a predicate symbol node.

For ease of algorithm definition, we introduce a new constraint (A-3').

(A-3') When generating an implication graph from input Horn clauses, the implication graph must be connected. If we have a non-connected graph for the implication graph, we are able to split the problem into two smalle problems which do not share unknown predicate symbols.

The constraint (A-4) remains valid in this subsection. This is equivalent to that the implication graph is acyclic.

Next, we present the traversing algorithm on the implication graph. It starts traversing from the linear inequality node without outgoing edges, and generates a set of implication trees, that are processed by the same algorithm as the previous section. Although we may visit the same predicate symbol multiple times, we handle it by renaming the predicate symbols differently every time and unify all those copies in the end. Shown below is the pseudo-code for the traversal.

<pseudo-code>
Input: Implication graph
Output: Mapping from pred symbol to a set of its copies, and a set of impl trees

Implication trees generated from the algorithm is processed by the same method in the previous section. We obtain a linear constraints of parameters and an parameterized assignment for each implication tree.

Finally, we explain about the unification. We duplicated the predicate symbols for every appearance during the traversal, and we notice that all of them are implied by the same hypothesis. Therefore, we create a conjunction of all parameterized assignments and let it be the new assignment for the original predicate symbol. By this method, we can merge several several solution spaces out of different implication trees into one solution space of the original problem.

Merging two solution spaces $(c_1,\theta_1)$ and $(c_2,\theta_2)$ into $(c, \theta)$ is done as follows.

$$P_1, P_2, ... \in Copies of P
c := c_1 /\ c_2
\theta := { [ x1 /\ x2 / P ] | [x1/P1], [x2/P2] \in \theta_1 \cup \theta_2 }$$

\section{Linear arithmetic with disjunction}

We assumed that all the linear inequalities were connected conjunctively. However, when we try to infer dependent types for the following SHP, it gives raise to the occurrence of disjunction.

<example of SHP>
<Horn clauses with disjunctions>

For this reason, we must extend our algorithm to support disjunctions in linear inequalities. The extended grammar syntax is shown below.

\setlength{\grammarindent}{2em}
\begin{grammar}
<la> ::= <expr> | $<la> \wedge <la>$ | $<la> \vee <la>$
\end{grammar}

By this change, our algorithm is affected for following points.

(TODO)
\begin{itemize}
\item Linear inequality node becomes a disjunction of linear inequalities.
\item For pred symbols on the path from such node with disjunction to the root, the assignment is connected by $\vee$
\item Other preds are connected by $\wedge$
\item First: Graph traversal to trees. Then: Choose one from disjunction.
	(TODO: Why the other way is NG?)
\end{itemize}

\section{Unification of different predicate symbols}
Our suggested algorithm only handles LA and not UIF. For this reason, we cannot treat Boolean values in predicate symbol parameters. For an intuitive example, the following Horn clause solving problem is not valid.

<Horn clauses with bool>

Therefore, we tried to work around this problem by extending Horn clauses with Boolean parameters as follows.

<expanded Horn clauses>

In this case, the original predicate symbol $P$ is passed to our algorithm as $P_T$ and $P_F$, and we want to obtain the same preposition for them if possible.

(TODO)


\subsection{Restriction-less Horn clauses}

TODO:
Unification of predicate symbols over expanded loop.
$P_1$, $P_2$, ...
Failure to unify means the absence of solution (complete)

citation from PPDP paper


\section{Completeness and Complexity}

?

\section{Experiment}

We limited to integer space

\section{Future Work}
\section{Conclusion}

\bibliographystyle{plain}
\bibliography{./biblio}

\end{document}
